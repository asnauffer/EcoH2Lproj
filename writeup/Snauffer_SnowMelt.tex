\documentclass[12pt]{article}
\usepackage{mathptmx} % use Times Roman font equivalent

\usepackage[doublespacing]{setspace} % double space the document

\usepackage[utf8]{inputenc}

\usepackage{graphicx}
\graphicspath{ {images/} }

\usepackage[export]{adjustbox}

\usepackage{float} % controls floating of images

\usepackage{color} % highlighting
\newcommand{\hilight}[1]{\colorbox{yellow}{#1}} % does not wrap

\usepackage{soul} % \hl does wrapping highlighting
% need to install separately:  sudo apt-get install texlive-latex-extra

\usepackage{amsmath} % typesetting of equations

\usepackage{makecell} % formats a cell (e.g. splitting text)

\usepackage{textcomp} % for symbols (like degrees)

\usepackage{natbib}
% \renewcommand*{\refname}{Bibliography}

\usepackage{listings}

\usepackage{url}

\usepackage{pdflscape}

\renewcommand\thetable{\Roman{table}} % use roman numerals in table

\hyphenation{ERA-Land ERA-Interim MERRA-Land}

\title{Modifying a Snow Model from Single to Dual Layer Design}
 
\author{}
\date{}

\pdfinfo{
  /Title    ()
  /Author   ()
  /Creator  ()
  /Producer ()
  /Subject  ()
  /Keywords ()
}

\begin{document}
\maketitle
\begin{center}
Andrew M. Snauffer (asnauffer@eos.ubc.ca)\textsuperscript{1*}
\newline
\newline
\footnotesize{
\textsuperscript{1}\textit{
Department of Earth, Ocean and Atmospheric Sciences,
The University of British Columbia,
Vancouver, BC V6T 1Z4,
Canada}
}
\end{center}

\begin{footnotesize}
\noindent
*Correspondence to: Andrew M. Snauffer, 
Department of Earth, Ocean and Atmospheric Sciences,
The University of British Columbia,
Vancouver, BC V6T 1Z4,
Canada
\\E-mail: asnauffer@eoas.ubc.ca
\\Mobile: +1-778-986-3319

\end{footnotesize}

\newpage

\section{Abstract}

\bigskip
\noindent KEY WORDS \: snow water equivalent; alpine regions; snow course; 

\section{Introduction}
snow water equivalent (SWE)

\section{Methods}
Full energy balance snow models require many data inputs that may not be easily accessible, while simple temperature-index models need substantial information and are subject to significant uncertainties in the calibration of the melt factor.
The SnowMelt function of the R package EcoHydRology \citep{fuka2014ecohydrology} addresses this challenge by estimating a portion of the data necessary to the energy balance such that the model works using no more data than a simple temperature-index model.
A full accounting of those approximations is given in \citet{walter2005process}.

A listing of the SnowMelt source code is presented Appendix A.
A summary of the function procedure follows.
\begin{enumerate}
  \item Set relevant constants (latent heat of vaporization and fusion, snow heat capacity, etc.)
  \item Convert inputs: mean daily temperature, daily snow based on precipitation and mean temperature, new snow density based on mean temperature \citep{goodison1981measurement}, resistance to heat transfer \citep{campbell1977introduction}, and cloudiness and atmospheric emissivity \citep{fuka2014ecohydrology}
  \item For time step 1:
  \begin{enumerate}
    \item Set starting SWE and snow depth using initial input values for snow depth and density.
    \item Calculate snow albedo based on ground albedo or amount of new snow, if new snow has been received.
    \item Calculate insolation based on latitude, Julian date, min and max daily temperature, albedo, forest cover, slope, and aspect.
    \item Calculate sensible heat exchanged based on mean daily temperature, snow temperature, and thermal resistance.
    \item Calculate vapor energy based on latent heat of vaporization, thermal resistance, and saturated vapor density at both snow and atmospheric temperatures.
    \item Calculate daily longwave radiation based on the Stefan-Boltzman equation.
    \item Calculate net energy based on insolation, atmospheric and terrestrial longwave radiation, sensible heat, vapor energy, ground conduction, and precipitation heat.
    \item Calculate snow density by blending existing and new snow.
    \item Calculate snowmelt based on net energy above that required to raise snow temperature to the melting point.
    \item Calculate SWE by summing changes (old + new - melt) and find resulting snow depth by dividing by density.
  \end{enumerate}
  \item Loop through remaining time steps:
  \begin{enumerate}
    \item Calculate snow albedo as above.
    \item Calculate insolation as above.
    \item If there is snow on the ground, set a density coefficient and recalculate snow temperature using the energy balance plus previous and current density and depth values.
      % 		SnowTemp[i] <- max(min(0,Tmin_C[i]),  # can't be 
      % 				   min(0,(SnowTemp[i-1]+
      % 					  min(-SnowTemp[i-1],
      % 	Energy[i-1]/((SnowDensity[i-1]*SnowDepth[i-1]+NewSnow[i]*NewSnowDensity[i])*SnowHeatCap*1000)))))
    \item Recalcalculate saturated vapor density at new snow temperature.
    \item Calculate sensible heat exchanged as above.
    \item Calculate vapor energy as above.
    \item Calculate daily longwave radiation as above.
    \item Calculate net energy and if positive, set a constant to increase snow density.
    \item Calculate snow density by blending existing and new snow, using the previous snow density adjusted such that it exponentially approaches 450.
      % 		if(SnowDepth[i-1]+NewSnow[i])>0) {
      % 			SnowDensity[i] <- min(450,
      % 					      ((SnowDensity[i-1]+k*30*(450-SnowDensity[i-1])*exp(-DCoef[i]))*SnowDepth[i-1] + NewSnowDensity[i]*NewSnow[i])/(SnowDepth[i-1]+NewSnow[i]))
      % 		} else {
      % 			SnowDensity[i] <- 450
      % 		}
    \item Calculate SWE by summing changes (old + new - melt) and find resulting snow depth by dividing by density.
  \end{enumerate}
\end{enumerate}

This approach uses a one-layer snowpack with properties uniformly set throughout.
However, the one-layer approach fails to take into account energy and melt variations within the snowpack, specifically the insulating properties of upper snow layers \citep{jost2012distributed}.
This previous work handled this limitation by constructing a two-layer snowpack with an upper layer of 100 mm SWE.
A similar approach was employed by the current study in order to improve the representation of the seasonal snow cycle.

A listing of the modified SnowMelt source code, named SnowMelt2L, is presented Appendix B.
A summary of the changes implemented is as follows.
\begin{enumerate}
  \item Check if snowpack meets one-layer condition (SWE $<$ 100 mm) prior to calculating snow temperature as before.
  \item Else if snowpack SWE $>$ 100 mm, recalculate previous upper and lower snow temperatures using the same approach with different energy inputs and components (e.g.\ ground conduction to lower layer, all other terms to upper layer).
  \begin{enumerate}
    \item If the new SWE $\geq$ 100 mm, \\
    new upper snow temperature = mean air temperature, and \\
    new lower snow temperature = a blend of new snow and the previous upper layer (accounting for snow temperature changes).
    \item If the new SWE $<$ 100 mm, \\
    new upper = a blend of new snow and part of the upper layer, and \\
    new lower = a blend of snow moved down from the upper layer and the previous lower layer.
  \end{enumerate}
  \item Calculate surface vapor density, sensible heat, vapor energy, and longwave radiation (rhos, H, E, and Lt) as before.
  \item Check again if snowpack meets one-layer condition (SWE $<$ 100 mm) prior to calculating energy, snow density, snowmelt, SWE, and snow depth as before.
  \item Else if snowpack SWE $>$ 100 mm, calculate the above terms using two layers.
  \begin{enumerate}
    \item Partition energy between layers (ground conduction to lower layer, everything else to upper). Note: this does not consider conduction across layers.
    \item Calculate snow density as before, blending new snow and existing snow density, which exponentially approaches 450.
    \item Calculate snowmelt, snow depth and SWE as before, using a maximum 100 mm upper layer and respective energy and snow temperature terms. Totals = sum of upper and lower. Note: this does not consider refreezing.
  \end{enumerate}
\end{enumerate}

An initial assessment of the SnowMelt2L performance revealed that the two-layer model was in many cases melting off snow earlier than the one-layer design.
Further examinations of the code showed that the selection of the albedo decay function was based on a snow depth threshold set to 0.1 m in the code.
This conditional was different from that specfied in \citep{walter2005process}, which specified a SWE threshold of 0.3 m water equivalent.
A subsequent change in the code was made to allow for the specification of a SWE threshold, which was implemented as follows.
\begin{enumerate}
  \item Accept an optional input argument to SnowMelt2L that allows a SWE threshold to be checked for selection of an albedo decay function.
  \item If there is new snow, set albedo as a function of the previous albedo and the amount of new snow.
  \item Else if no albedo threshold is passed and the previous snow depth $<$ 0.1 m, decay the albedo linearly with time.
  \item Else if an albedo threshold is passed and previous SWE $<$ that threshold, set the albedo as a function of SWE.
  \item Else decay the albedo with time according to the empirical relationship developed by \citet{armycorps1960runoff}.
\end{enumerate}

A driver code for the SnowMelt model was also written.
This code extracted snow pillow station information, performed error checking on the data, and plotted yearly time series for measured and modeled values.
The driver code, called aspEcobase.R, is shown in Appendix C.



\section{Results}



\section{Discussion}



\section{Conclusion}



\section{Acknowledgements}
Funding for this work was provided by the Canadian Sea Ice and Snow Evolution (CanSISE) Network funded through the Climate Change and Atmospheric Research (CCAR) initiative of Canada's Natural Sciences and Engineering Research Council (NSERC).  


\bibliographystyle{natbib}
\bibliography{Snauffer_biblio}

\begin{landscape}
\pagestyle{empty}

\lstset{
  basicstyle=\linespread{0.75}\footnotesize,
  tabsize=2,
  breaklines=true,
  numbers=left
}

\section{Appendix A}
The following is the SnowMelt source code, part of the EcoHydRology R package version 0.4.12, published April 4, 2014.
It was obtained from \url{https://cran.r-project.org/web/packages/EcoHydRology/index.html} on January 9, 2018.

\bigskip

\lstinputlisting{../SnowMelt.R}

\newpage
\section{Appendix B}
The following is the SnowMelt2L source code, the modified code developed in this work.

\bigskip

\lstinputlisting{../SnowMelt2L.R}

\newpage
\section{Appendix C}
The following is the aspEco\_base.R source code, the driver code developed in this work which loads, executes and plots automated snow pillow (ASP) data and snow model results.

\bigskip

\lstinputlisting{../aspEco_base.r}


\end{landscape}

\end{document}


  